\documentclass[xcolor=table]{beamer}
\mode<presentation>

\usepackage{ulem}
\usepackage[T1]{fontenc}
\usepackage[scaled=0.85]{beramono}
\usepackage{framed}
\usepackage{listings}


\usepackage{tikz}
\usetikzlibrary{shapes}
\usetikzlibrary{positioning,arrows,shadows,decorations.pathreplacing}

\setbeamertemplate{navigation symbols}{}
\usetheme{Boadilla}
\usecolortheme{beaver}
\usefonttheme{serif}

\linespread{1.5}

\title{Quantatative and Qualatative Qualities of Contemporary Capability Constructions}
\date{Feb 26, 2016}
\author{Kritika Iyer, Sajin Sasy, Justin Tracey}

\begin{document}

\begin{frame}
  \titlepage
  \centering
  CS 854
\end{frame}

\begin{frame}
  \frametitle{Access Control}
  \begin{itemize}
  \item Systems, including Operating Systems, control resources.
  \item How do they designate who gets access to what resource?
  \item Answer: Access Control Tables
  \end{itemize}
\end{frame}

\begin{frame}
  \frametitle{Example Access Control Table}
  \begin{center}
    \begin{tabular}{|l||c|c|c|}
      \hline
      &File 1&File 2&File 3\\
      \hline
      \hline
      Alice&rw&rx&rwx\\
      \hline
      Bob&&rx&\\
      \hline
      Carol&r&rx&\\
      \hline
    \end{tabular}\\
    \vspace{10pt}
    \invisible{Line here for consistent spacing}
  \end{center}
\end{frame}

\begin{frame}
  \frametitle{Access Control Representation}
  How is this table actually represented?
  \begin{itemize}
  \item Representing the entire table is wasteful... \pause
  \item Access Control Lists
  \item Capabilities
  \end{itemize}
\end{frame}

\begin{frame}
  \frametitle{Access Control Lists}
  \begin{center}
    \begin{tabular}{|l||c|c|c|}
      \hline
      &\cellcolor{green}File 1&File 2&File 3\\
      \hline
      \hline
      Alice&\cellcolor{green}rw&rx&rwx\\
      \hline
      Bob&\cellcolor{green}&rx&\\
      \hline
      Carol&\cellcolor{green}r&rx&\\
      \hline
    \end{tabular}\\
    \vspace{10pt}
    ``I am File 1. Alice has rw permissions, Carol has r permissions.''
  \end{center}
\end{frame}

\begin{frame}
  \frametitle{Capabilities}
  \begin{center}
    \begin{tabular}{|l||c|c|c|}
      \hline
      &File 1&File 2&File 3\\
      \hline
      \hline
      Alice&rw&rx&rwx\\
      \hline
      Bob&&rx&\\
      \hline
      \cellcolor{green}Carol&\cellcolor{green}r&\cellcolor{green}rx&\cellcolor{green}\\
      \hline
    \end{tabular}\\
    \vspace{10pt}
    ``I am Carol, with r permissions for File 1, and rx permissions for file 2.''
  \end{center}
\end{frame}

\begin{frame}
  \frametitle{The 4 Types of Capability Systems}
  
\begin{enumerate}
    \item{Tagged with bits}
    \item{Tagged with type system}
    \item{Segregated}
    \item{Password/ Sparse}
\end{enumerate}
 \end{frame}

\begin{frame}{Object-capability Programming Languages}
Programming languages to implement capability security:
\begin{itemize}
    \item Joule
    \item E
    \item Joe-E
    \item Caja project
\end{itemize}
\end{frame}

\begin{frame}
  \frametitle{Open Problems}
\end{frame}

\begin{frame}
  \frametitle{Our Goals}
\end{frame}

\end{document}